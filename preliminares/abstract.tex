\chapter{Abstract}

\textbf{Abstract:} 
The increasing demand for robust and efficient information security has led to the growing adoption of specialized hardware for cryptographic operations. In response to the rise in cyber threats and the need to process large volumes of data in real time, hardware-based cryptographic solutions offer significant advantages in terms of performance, resistance to attacks, and secure storage of cryptographic keys.

This thesis presents the implementation of a secure communication system using the UART (Universal Asynchronous Receiver-Transmitter) protocol as the foundation for a Register Transfer Level (RTL) design on an FPGA platform. The base protocol was modified to introduce an additional hardware-level security layer. Furthermore, cryptographic techniques—specifically encryption and decryption—were integrated into the design to enhance data protection and integrity during transmission. The results demonstrate the feasibility of embedding cryptographic mechanisms directly into communication hardware, providing a scalable and efficient solution for secure embedded systems.





\textbf{Palabras Clave:} Cryptographic hardware, Information security, Encryption, Decryption, Cryptographic keys, Digital signatures, Authentication, Cyber threats, Data processing, Communication protocols, UART (Universal Asynchronous Receiver-Transmitter), FPGA (Field-Programmable Gate Array), RTL design (Register Transfer Level), Hardware security layer, Cryptographic techniques, Critical infrastructure.\\


%%%%%%fin del archivo
\endinput 