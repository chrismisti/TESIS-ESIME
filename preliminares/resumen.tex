\chapter{Resumen}


\textbf{Resumen:}

La creciente demanda de seguridad inform\'atica robusta y eficiente ha impulsado la adopci\'on de hardware especializado para operaciones criptogr\'aficas. Ante el aumento de las amenazas cibern\'eticas y la necesidad de procesar grandes vol\'umenes de datos en tiempo real, las soluciones criptogr\'aficas basadas en hardware ofrecen ventajas significativas en t\'erminos de rendimiento, resistencia a ataques y almacenamiento seguro de claves criptogr\'aficas.

Esta tesis presenta la implementaci\'on de un sistema de comunicaci\'on segura utilizando el protocolo UART (Universal Asynchronous Receiver-Transmitter) como base para un diseño a nivel de transferencia de registros (RTL) sobre una plataforma FPGA. El protocolo base fue modificado para introducir una capa adicional de seguridad a nivel de hardware. Adem\'as, se integraron t\'ecnicas criptogr\'aficas espec\'ificamente cifrado y descifrado en el diseño con el objetivo de mejorar la protecci\'on e integridad de los datos durante la transmisi\'on. Los resultados demuestran la viabilidad de incorporar mecanismos criptogr\'aficos directamente en el hardware de comunicaci\'on, proporcionando una soluci\'on escalable y eficiente para sistemas embebidos seguros.



\textbf{Palabras Clave:} Hardware criptográ\'afico, Seguridad inform\'atica, Cifrado, Descifrado, Claves criptogr\'aficas, Firmas digitales, Autenticaci\'on, Amenazas cibern\'eticas, Procesamiento de datos, Protocolos de comunicaci\'on, UART (Universal Asynchronous Receiver-Transmitter), FPGA (Field-Programmable Gate Array), Diseño RTL (Register Transfer Level), Capa de seguridad en hardware, T\'ecnicas criptogr\'aficas, Infraestructuras cr\'iticas. \\



%%%%%%fin del archivo

\endinput 