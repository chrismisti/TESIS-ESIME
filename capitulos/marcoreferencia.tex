\chapter{Marco Teórico Contextual}

\section{Marco Teórico Contextual}
Para contextualizar adecuadamente el presente trabajo de tesis dentro del marco teórico, es necesario abordar el origen del algoritmo criptográfico estándar César, así como su evolución y aplicaciones a lo largo del tiempo. Asimismo, se explorarán las condiciones en las que surgieron los servicios y mecanismos de seguridad informática, y las formas en que estos han sido implementados en distintos entornos. Este análisis permite delimitar la problemática que se busca atender mediante la propuesta de implementación en hardware presentada en esta investigación. En consecuencia, se realiza una revisión del estado del arte en la disciplina, con especial énfasis en los sistemas embebidos que incorporan mecanismos de seguridad.

A través de esta revisión se busca comprender las bases que fundamentan el diseño de soluciones criptográficas en plataformas reconfigurables, como las FPGAs, y su relevancia en escenarios donde la protección de la información es crítica. Este marco teórico sienta las bases para el desarrollo metodológico y experimental abordado en los siguientes capítulos.

\section{Fundamentos de la Criptografía clasica}
La criptografía es la ciencia y arte de proteger la información mediante técnicas que transforman datos legibles (texto plano) en datos ininteligibles (texto cifrado), de modo que sólo las personas autorizadas puedan acceder a su contenido (Stallings, 2017). A lo largo de la historia, la criptografía ha evolucionado desde métodos simples de sustitución y transposición hasta complejos algoritmos basados en teoría matemática, álgebra moderna y teoría de números.

Los métodos clásicos, aunque hoy son inseguros frente a los ataques modernos, permiten comprender los conceptos fundamentales sobre el uso de claves, cifrado simétrico y análisis criptográfico. Uno de los algoritmos más representativos de la criptografía clásica es el cifrado César, también conocido como cifrado por desplazamiento.

\section{El Cifrado César: Origen Histórico}
El cifrado César recibe su nombre de Julio César, quien lo empleaba para cifrar mensajes militares. Según registros históricos, César utilizaba un desplazamiento de tres posiciones para codificar sus mensajes (Kahn, 1996). Aunque su simplicidad hoy lo convierte en un cifrado trivial, en su época ofrecía una protección básica frente a lectores no entrenados o no alfabetizados.

Este cifrado pertenece a la familia de los cifrados monoalfabéticos por sustitución, donde cada letra del alfabeto es reemplazada por otra, según una regla fija determinada por una clave de desplazamiento.

\section{Funcionamiento del Algoritmo}

Desde un punto de vista matemátoco, el cifrado César puede representarse mediante la siguiente función:

\centerline{\(C(x) = ( x + k ) mod n \)}

Donde:

\begin{itemize}
\item \(C(x)\) es el carácter cifrado.
\item \(x\) es el índice del carácter en el alfabeto \( A = 0, B = 1, ...., Z = 25\)
\item \(k \) es la clave de desplazamiento.
\item \(n\) el es número total de caracteres del alfabeto, usualmente  \(n = 26 \) en el alfabeto latino.
\end{itemize}

El descifrado consiste en aplicar la operación inversa 

\centerline{\(P(x) = ( x - k ) mod n \)}

Por ejemplo, si se cifra la palabra \textbf{\textit{''HOLA''}} con un desplazamiento de \(k = 3\) se obtiene \textbf{\textit{''KROD''}} . Para descifrar, se aplica el desplazamiento inverso. 

\section{Seguridad y Vulnerabilidades}

El cifrado César es vulnerable a varios tipos de ataques criptográficos:

\begin{itemize}
\item Ataque por fuerza bruta: Dado que existen solamente 25 posibles claves (excluyendo el desplazamiento nulo), un atacante puede probar todas las combinaciones en poco tiempo.
\item Análisis de frecuencia: Cada idioma tiene una distribución característica de letras. En español, por   las letras ''E'', ''A'' y ''O'' son las más comunes. Si se cifra un texto suficientemente largo, estas frecuencias se preservan, permitiendo identificar el desplazamiento utilizado.
\end{itemize}

Estas debilidades lo hacen inapropiado para cualquier uso moderno que requiera confidencialidad real, pero útil como herramienta de desarrollo.

\section{Utilidad en el Diseño de Sistemas Criptográficos}
A pesar de sus limitaciones, el cifrado César es frecuentemente utilizado en el diseño de sistemas criptográficos hardware/software con fines educativos y experimentales. Su estructura simple lo convierte en una opción ideal para:

\begin{itemize}
\item Introducir técnicas de procesamiento secuencial de datos.
\item Implementar máquinas de estados finitos para codificación y decodificación.
\item Evaluar el consumo de recursos lógicos y tiempos de propagación en plataformas como FPGAs.
\item Comparar el comportamiento de cifrados más complejos respecto a algoritmos básicos en entornos de bajo nivel (VHDL, Verilog).
\end{itemize}

En proyectos de diseño digital con arquitectura RTL (Register Transfer Level), el cifrado César se utiliza como caso de estudio para optimizar operaciones de desplazamiento modular, codificación de caracteres ASCII y generación de claves.



\chapter{Antecedentes System Verilog}

\section{Evolución del diseño digital y el surgimiento de SystemVerilog}
El desarrollo del diseño digital ha experimentado transformaciones significativas desde sus inicios, comenzando con el prototipado físico mediante placas de pruebas (protoboards) hasta llegar a sofisticadas plataformas de simulación y modelado asistido por computadora. Herramientas CAD (Computer-Aided Design) permitieron simular circuitos electrónicos con mayor precisión y eficiencia, aunque su uso inicial requería superar ciertas barreras de complejidad técnica.

Estas herramientas facilitaron la implementación de esquemas eléctricos y permitieron definir señales de entrada coherentes con la lógica del sistema, generando salidas fácilmente interpretables. Simuladores como SPICE se convirtieron en referentes dentro del ámbito académico e industrial, al ofrecer una modelación precisa del comportamiento eléctrico. Una vez validados los resultados, se procedía al diseño del circuito físico mediante herramientas especializadas en tarjetas de circuito impreso (PCB), como el entorno OrCAD, particularmente en su módulo PCB Layout.

Durante esta etapa, el Laboratorio de Computación Adaptable comenzó a implementar modelos de circuitos neuronales electrónicos mediante estas plataformas (Padrón, 1997). Posteriormente, ante el incremento en la complejidad de los modelos de redes neuronales, se incorporaron herramientas como MATLAB y su entorno gráfico SIMULINK, permitiendo una representación matemática directa de sistemas dinámicos. Esta evolución posibilitó una interpretación más eficaz de los resultados, tanto cuantitativa como cualitativamente (Padrón et al., 2000, pp. 338–349).

De forma paralela, la industria del hardware reconfigurable, especialmente a través de las tarjetas FPGA manufacturadas por Xilinx, permitió abordar problemas específicos mediante interfaces programables. Cada tarjeta está diseñada para ofrecer flexibilidad al usuario, quien puede programar sus componentes en lenguajes de descripción de hardware como Verilog o VHDL, e integrar diversas interfaces en función de la disponibilidad física o lógica del sistema. Esta versatilidad convirtió a los kits de desarrollo en herramientas fundamentales para el diseño de soluciones a medida, especialmente en contextos de prototipado rápido y validación funcional.

Conforme se incrementaron las necesidades de diseño a nivel de sistemas y la verificación se volvió una etapa crítica, surgió SystemVerilog como una extensión y evolución del lenguaje Verilog. SystemVerilog no solo incorporó mejoras en la descripción estructural y comportamental del hardware, sino que integró capacidades orientadas a la verificación funcional, programación orientada a objetos y modelado de sistemas complejos. Esta evolución respondió a la necesidad de unificar el flujo de diseño y verificación bajo un mismo lenguaje, facilitando la implementación de testbenches avanzados, interfaces reutilizables y entornos de verificación compatibles con metodologías modernas como UVM (Universal Verification Methodology).

En este sentido, el uso de tarjetas FPGA programadas con SystemVerilog permite combinar la precisión del diseño RTL con capacidades avanzadas de verificación, lo cual resulta esencial en entornos de desarrollo donde se busca garantizar funcionalidad, rendimiento y confiabilidad desde las etapas tempranas del proyecto. La historia y evolución de este lenguaje refleja una tendencia hacia la consolidación de herramientas que integren diseño, simulación, verificación y síntesis dentro de un mismo entorno de desarrollo.

Cronología relevante del desarrollo de SystemVerilog 
\begin{itemize}
\item 1984: Se introduce el lenguaje Verilog como herramienta para descripción de hardware por Gateway Design Automation, posteriormente adquirido por Cadence (IEEE, 2008).
\item 1990s: Verilog se convierte en uno de los estándares principales para diseño digital, ampliamente utilizado en la industria electrónica.
\item 1999: Se inicia el desarrollo de SystemVerilog como una extensión de Verilog para abordar las limitaciones en verificación y modelado (Accellera Systems Initiative, 2002).
\item 2002: SystemVerilog es adoptado formalmente por Accellera como estándar para diseño y verificación.
\item 2005: SystemVerilog es estandarizado por IEEE como el estándar IEEE 1800-2005, consolidando su uso en la industria (IEEE, 2005).
\item 2012: Se publica la revisión IEEE 1800-2012, incorporando mejoras en síntesis y verificación.
\item 2017: Nueva revisión IEEE 1800-2017 con ampliaciones en características para diseño y verificación de sistemas complejos.
\end{itemize}

\section{Lenguajes de Descripción de Hardware (HDL) y su Aplicación en el Diseño Digital}

Los lenguajes de descripción de hardware (HDL, por sus siglas en inglés) surgieron como respuesta a la necesidad de los diseñadores digitales de contar con herramientas formales que permitieran especificar, modelar y verificar sistemas digitales de forma estructurada y en distintos niveles de abstracción. Estos lenguajes no solo facilitan la comunicación entre diseñadores, sino que también permiten una interacción fluida entre las herramientas de diseño asistido por computadora (CAD) y los propios modelos digitales (Terés et al., 1998).

Los entornos de desarrollo basados en HDL integran herramientas de compilación, simulación y síntesis, lo que permite validar el comportamiento funcional de un diseño antes de su implementación física. En la actualidad, los lenguajes HDL más utilizados son VHDL y Verilog, ambos estandarizados por el IEEE (Institute of Electrical and Electronics Engineers), y ampliamente adoptados en la industria del diseño digital.

VHDL: Descripción, Aplicación y Estructura
VHDL (VHSIC Hardware Description Language) fue desarrollado originalmente para documentar y verificar circuitos integrados de alta velocidad dentro del programa VHSIC del Departamento de Defensa de los EE.UU. Su sintaxis y semántica están basadas en el lenguaje ADA, lo cual le proporciona una estructura sólida, orientada a sistemas críticos y de tiempo real.

VHDL permite modelar un sistema digital a través de diferentes niveles de descripción: comportamiento, transferencia de registros (RTL) y nivel lógico estructural. Esto permite abarcar prácticamente todo el ciclo de desarrollo digital, desde las especificaciones funcionales hasta la generación del prototipo, excluyendo únicamente el trazado físico o layout.

La correcta elección del nivel de abstracción en la descripción depende del objetivo de diseño. Por ejemplo, si se busca generar una implementación física mediante herramientas de síntesis, es preferible utilizar el nivel RTL. En cambio, para validar la funcionalidad de algoritmos complejos mediante simulación, el nivel algorítmico es más adecuado (Baena, 2010).


\section{SystemVerilog: Lenguaje de Descripción y Verificación de Hardware }

SystemVerilog es un lenguaje de descripción de hardware (HDL) y verificación funcional desarrollado como una extensión del lenguaje Verilog, con el objetivo de unificar el modelado estructural, la verificación orientada a objetos y la abstracción de sistemas en un solo entorno. Su aparición responde a la necesidad creciente de los diseñadores digitales de contar con herramientas capaces de describir, validar y verificar sistemas digitales complejos, como SoCs (System-on-Chip), en múltiples niveles de abstracción.

A diferencia de VHDL, SystemVerilog incorpora elementos sintácticos y semánticos tanto de lenguajes de descripción como de programación (inspirado en C y C++), lo que lo convierte en un lenguaje híbrido y altamente expresivo. Fue estandarizado por el IEEE como el estándar IEEE 1800, inicialmente en 2005, y ha sido adoptado ampliamente en la industria por su compatibilidad con flujos de diseño RTL y metodologías de verificación como UVM (Universal Verification Methodology).

Enfoque de Verificación en SystemVerilog
Una de las mayores fortalezas de SystemVerilog es su orientación a la verificación, lo que lo diferencia fundamentalmente de lenguajes como VHDL o Verilog puro. 

\section{Comparativa entre Lenguajes de Descripción de Hardware y Desarrollo de Software}
Desde sus orígenes, las distintas herramientas y entornos de diseño asistido por computadora (CAD) han empleado lenguajes y formatos específicos para representar y gestionar los diversos elementos involucrados en el diseño electrónico. Algunos de estos lenguajes consistían en notaciones explícitas utilizadas por el usuario para describir entradas a determinadas herramientas, mientras que otros correspondían a formatos intermedios internos, optimizados para el procesamiento automatizado dentro del entorno CAD y generalmente ocultos al diseñador.

Sin embargo, estas descripciones se encontraban limitadas al ecosistema particular de cada herramienta, sin ofrecer portabilidad ni estandarización. La aparición de los lenguajes de descripción de hardware (HDL, por sus siglas en inglés) supuso un avance significativo al introducir un mayor grado de estandarización y al incorporar principios propios de la ingeniería de software en la especificación y modelado de hardware digital.

Desde el punto de vista sintáctico, los HDL presentan similitudes con los lenguajes de programación de alto nivel (HLL), lo cual facilita su aprendizaje por parte de ingenieros con experiencia previa en desarrollo de software. Por ejemplo, Verilog comparte muchas características con el lenguaje C, mientras que VHDL hereda su estructura y semántica del lenguaje ADA. Estas semejanzas, si bien pueden acelerar la adopción de HDL, pueden también inducir errores conceptuales si se emplean estrategias propias del software en contextos donde se requiere una representación precisa del comportamiento físico del hardware.

Es fundamental que, cuando un modelo HDL esté destinado a la síntesis e implementación física (por ejemplo, en FPGAs o ASICs), el diseñador adopte una mentalidad orientada al hardware, describiendo la lógica con un enfoque estructural o de transferencia de registros (RTL), y no meramente algorítmico. Por otro lado, el uso de estructuras típicas del software puede ser apropiado cuando el objetivo sea únicamente la simulación funcional del sistema, permitiendo optimizar el rendimiento en las etapas de verificación y análisis temporal.

En resumen, los HDL son lenguajes formales de alto nivel, diseñados específicamente para representar circuitos electrónicos a diversos niveles de abstracción —desde compuertas lógicas básicas hasta sistemas digitales completos— y permiten un modelado flexible, estructurado y preciso del hardware, aprovechando conceptos tanto del diseño electrónico como del desarrollo de software, tal y como se muestran en la Figura \ref{fig:imagen1}. 
\begin{figure}[h!] % [h!] Fuerza a LaTeX a poner la figura aquí
    \centering % Centra la imagen
     \includegraphics[width=0.5\textwidth, height=6cm]{imagenes/img1} % Reemplaza con el nombre de tu archivo de imagen
    \caption{Niveles de abstracción/precisión y estilos de modelado VHDL.}
    \label{fig:imagen1} % Etiqueta para referenciar la figura
\end{figure} 
%\clearpage

Los lenguajes de descripción de hardware (HDL) fueron desarrollados inicialmente con el propósito de modelar el comportamiento funcional de los componentes electrónicos, permitiendo su simulación previa a la implementación física. No obstante, también son ampliamente utilizados para la descripción estructural de circuitos digitales, facilitando su posterior síntesis y verificación mediante procesos de simulación validados \ref{fig:imagen2}.
\begin{figure}[h!] % [h!] Fuerza a LaTeX a poner la figura aquí
    \centering % Centra la imagen
     \includegraphics[width=0.8\textwidth, height=6cm]{imagenes/img2} % Reemplaza con el nombre de tu archivo de imagen
    \caption{Desarrollo en software versus hardware: (a) niveles de abstracción y lenguajes de alto nivel y (b) esquema básico del diseño descendente con HDL.  .}
    \label{fig:imagen2} % Etiqueta para referenciar la figura
\end{figure} 

A partir de las descripciones iniciales, los enfoques de diseño descendente (top-down) permiten aplicar procesos progresivos de síntesis que conducen gradualmente a niveles más detallados de implementación, hasta alcanzar una descripción física específica y dependiente de la tecnología utilizada. La validación en cada etapa del diseño se lleva a cabo mediante simulaciones y análisis funcionales, permitiendo iteraciones sucesivas de verificación y corrección hasta lograr un comportamiento conforme a los requisitos especificados \ref{fig:imagen2}(b).


\section{Descripción de la tarjeta de desarollo Altera DE2-115 }

La tarjeta Altera DE2-115, desarrollada por Terasic Technologies, es una plataforma de prototipado basada en la FPGA Cyclone IV EP4CE115F29C7N de Altera (ahora Intel), orientada a entornos educativos, de investigación y desarrollo de sistemas digitales avanzados. Esta tarjeta ofrece una arquitectura versátil que permite implementar desde diseños lógicos básicos hasta sistemas digitales complejos, incluyendo sistemas embebidos, controladores personalizados, procesamiento de señales digitales (DSP) y prototipos de SoC (System-on-Chip).

Entre sus características principales se destaca la presencia de 114,480 elementos lógicos (LEs), 3.888 Kbits de memoria RAM embebida, y 528 pines de entrada/salida de propósito general (GPIO), lo que proporciona una gran flexibilidad para interactuar con múltiples dispositivos periféricos. La tarjeta también incluye memorias externas como SDRAM de 128 MB, SRAM de 2 MB, y Flash NOR de 2 MB, además de soporte para almacenamiento mediante tarjeta SD.

La DE2-115 incorpora interfaces esenciales para el desarrollo de sistemas interactivos, tales como puertos USB, Ethernet Gigabit, salida VGA, entradas y salidas de audio, además de un conjunto de dispositivos de entrada/salida integrados como interruptores, botones, LEDs y displays de siete segmentos. Adicionalmente, cuenta con un oscilador de cristal de 50 MHz y conectores de expansión compatibles con módulos adicionales.

Esta plataforma es ampliamente utilizada en laboratorios académicos y en entornos de investigación aplicada debido a su compatibilidad con herramientas de desarrollo como Quartus Prime, facilitando la implementación de diseños en lenguajes HDL como VHDL y SystemVerilog. Su capacidad para realizar simulación, verificación y síntesis de diseños en una sola herramienta la hace ideal para proyectos de educación superior en ingeniería electrónica, mecatrónica y sistemas digitales. 

Características técnicas destacadas:

FPGA: Altera Cyclone IV EP4CE115F29C7N con 114,480 elementos lógicos (LEs)

\begin{itemize}
	\item Memoria:
		\begin{itemize}
			\item 2 MB SRAM
			\item 128 MB SDRAM
			\item 2 MB Flash NOR
			\item Tarjeta SD para almacenamiento externo
		\end{itemize}
	\item Interfaces de usuario:
		\begin{itemize}
			\item 18 interruptores (switches) y 18 botones pulsadores
			\item 18 LEDs rojos y 9 LEDs verdes
			\item 8 displays de 7 segmentos
		\end{itemize}
	\item Conectividad:
		\begin{itemize}
			\item Puertos USB tipo A y B
			\item Puerto Ethernet 10/100/1000 Mbps (Gigabit)
			\item Entrada/salida VGA
			\item Conectores de expansión GPIO de 40 pines
		\end{itemize}
	\item Audio y Video:
		\begin{itemize}
			\item Entrada y salida de audio (jack de 3.5 mm)
			\item Entrada de señal VGA (con ADC) y salida VGA
		\end{itemize}
	\item Reloj: oscilador de cristal de 50 MHz
	\item Programación y depuración: Soporte JTAG, compatible con Quartus II
\end{itemize}
 
\begin{figure}[h!] % [h!] Fuerza a LaTeX a poner la figura aquí
    \centering % Centra la imagen
     \includegraphics[width=1\textwidth, height=13cm]{imagenes/img3} % Reemplaza con el nombre de tu archivo de imagen
    \caption{Tarjeta de desarollo Altera DE2-115.}
    \label{fig:imagen3} % Etiqueta para referenciar la figura
\end{figure}

%\clearpage


\chapter{Diseño de Plataforma para Transmisión Serial de Datos}

\section{Plataforma de Comunicación Serial para el Cifrado y Descifrado }
La seguridad de la información, especialmente en lo que respecta a la confidencialidad, representa una de las principales preocupaciones en el ámbito de las comunicaciones. Para mitigar los riesgos de acceso no autorizado, es común la implementación de algoritmos criptográficos que aseguren que solo el receptor legítimo pueda interpretar los datos transmitidos.

Uno de los principales objetivos de este trabajo es desarrollar una plataforma basada en FPGA que permita integrar mecanismos de codificación de flujo para proteger la transmisión de datos entre dos dispositivos que se comunican mediante el protocolo UART. En lugar de utilizar UART en su forma tradicional, este protocolo es modificado y ampliado para incorporar una capa de seguridad directamente a nivel de hardware, mejorando así la confidencialidad de la información transmitida.

Para ello, se diseñó una plataforma base que incluye los módulos necesarios para enviar y recibir datos, realizar la codificación y decodificación de los mensajes, y establecer la comunicación segura entre ambos extremos. Esta arquitectura ha sido planteada de forma flexible, permitiendo que el proceso de codificación se defina de manera genérica. De este modo, es posible integrar diferentes algoritmos criptográficos sin necesidad de rediseñar por completo el sistema, lo que facilita su adaptación a distintos niveles de seguridad o aplicaciones específicas.

Un aspecto fundamental para el desempeño de un codificador de flujo es que su velocidad de operación no afecte la eficiencia de la comunicación. Por esta razón, la plataforma está implementada en una FPGA, lo que permite ampliar y modificar el protocolo de comunicación para incluir una capa de seguridad directamente a nivel hardware. Así, el proceso de cifrado se realiza en tiempo real, garantizando que la protección de los datos no genere retrasos ni impacte negativamente en la velocidad de transmisión.

Se realiza la prueba y validación del desempeño de la plataforma mediante la comparación de su funcionamiento al implementar un protocolo personalizado de comunicacion para incrementar el grado de seguridad.

\section{Protocolo Seguro para Comunicación Serial}
Como parte del desarrollo de esta tesis, se aborda el diseño de un sistema de comunicación seguro, iniciando con el análisis de las características fundamentales de los sistemas de comunicación digital y los mecanismos para garantizar que la información transmitida sea recibida y correctamente interpretada únicamente por el destinatario autorizado. Durante este proceso analítico se identifican múltiples criterios que definen las propiedades esenciales para un algoritmo criptográfico eficiente, los cuales serán abordados detalladamente en este capítulo.

En el proceso de establecer los parámetros para una comunicación segura, se busca lograr este objetivo minimizando el impacto sobre los sistemas de comunicación existentes. Cualquier mecanismo implementado para garantizar la seguridad debe ser transparente para el usuario y operar en tiempo real, sin afectar las características esenciales de la comunicación, tales como la calidad y la velocidad de transmisión.

Actualmente, la mayoría de los sistemas de comunicación emplean transmisión de datos serial, generalmente en modo full-duplex, para el intercambio de información. En la búsqueda de sistemas de comunicación seguros, funcionales y eficientes, es necesario analizar y evaluar en tiempo real la ejecución de un algoritmo criptográfico estándar. Posteriormente, será posible implementar y evaluar otros algoritmos conforme a las necesidades del sistema.

A partir de los lineamientos previamente establecidos, se definen las características fundamentales para el diseño de la plataforma de comunicación criptográfica, las cuales son:



\begin{enumerate}
\renewcommand{\theenumi}{\alph{enumi}}
\renewcommand{\labelenumi}{{\theenumi})}
\item Integración transparente: El sistema de cifrado debe ser incorporado en la línea de comunicación sin necesidad de modificar el equipo de comunicaciones existente, manteniendo así la compatibilidad con la infraestructura actual.

\item Compatibilidad con comunicación serial full-duplex asíncrona: La plataforma debe ser capaz de manejar flujos de datos bidireccionales simultáneos (full-duplex) en modo asíncrono, soportando múltiples velocidades de transmisión comúnmente utilizadas en entornos industriales o embebidos.

\item Flexibilidad criptográfica: Debe ofrecer la posibilidad de cambiar el algoritmo de cifrado de manera sencilla, permitiendo al usuario seleccionar entre distintas opciones según los requerimientos de seguridad del sistema.

\item Procesamiento en tiempo real: Las operaciones de cifrado y descifrado deben ejecutarse de forma continua y en tiempo real, sin interrupciones en el flujo de información, garantizando la integridad y la eficiencia de la comunicación.
\end{enumerate} 

Para alcanzar el primer objetivo —la integración transparente del sistema de seguridad— se diseñaron módulos hardware que pueden ser instalados en ambos extremos de la línea de comunicación. Estos dispositivos pueden colocarse directamente en los puntos finales del canal de transmisión, o bien actuar como una interfaz entre el usuario y el equipo de comunicación. La ubicación óptima dependerá del contexto de uso, de las características del equipo involucrado y del medio físico de transmisión utilizado.

La Figura \ref{fig:imagen4} muestra un diagrama de bloques que ilustra cómo puede integrarse el sistema de seguridad dentro de una arquitectura de comunicación serial, permitiendo el cifrado de los datos sin alterar la funcionalidad básica del canal.

\begin{figure}[h!] % [h!] Fuerza a LaTeX a poner la figura aquí
    \centering % Centra la imagen
     \includegraphics[width=1\textwidth, height=8cm]{imagenes/img4} % Reemplaza con el nombre de tu archivo de imagen
    \caption{Diagrama de bloques de un sistema de comunicación simplificado. a) Sistema Inseguro. 
b) Sistema con el cifrado en los extremos de la línea de la transmisión. c) Sistema con el cifrado 
como interfaz entre el usuario y el  equipo de comunicación. }
    \label{fig:imagen4} % Etiqueta para referenciar la figura
\end{figure} 

El mecanismo encargado del cifrado debe incluir, en términos generales, un módulo central de control responsable de coordinar todo el proceso de cifrado y descifrado. Este módulo gestiona el flujo de datos entre dos interfaces de comunicación serial, actuando como intermediario entre el transmisor y el receptor. Su funcionamiento se basa en controlar la entrada y salida de datos cifrados y descifrados de acuerdo con la lógica definida por el algoritmo criptográfico. La Figura \ref{fig:imagen5} ilustra la estructura general del sistema, destacando el papel del controlador central dentro del esquema de comunicación segura.

\begin{figure}[h!] % [h!] Fuerza a LaTeX a poner la figura aquí
    \centering % Centra la imagen
     \includegraphics[width=0.8\textwidth, height=3cm]{imagenes/img5} % Reemplaza con el nombre de tu archivo de imagen
    \caption{Diagrama de bloques del dispositivo de cifrado y Descifrado.}
    \label{fig:imagen5} % Etiqueta para referenciar la figura
\end{figure} 

Con el objetivo de garantizar que el proceso de cifrado y descifrado se ejecute de forma continua, sin interrumpir ni degradar el flujo original de información, se opta por el uso de algoritmos de cifrado por flujo. Este tipo de algoritmos permite el procesamiento de los datos en tiempo real, bit a bit o byte a byte, lo cual es especialmente adecuado para sistemas de comunicación serial donde la información se transmite de manera secuencial.






\section{Trabajos Relacionados}
En investigaciones previas se han propuesto mecanismos de seguridad sobre UART, pero en su mayoría dependen de software, lo que limita la eficiencia y vulnerabilidad ante ataques. Otros estudios han demostrado que la implementación de algoritmos criptográficos en FPGA, como AES en modo CTR, es factible y efectiva para mejorar la seguridad sin afectar el rendimiento significativamente.



\section{Metodología}
\subsection{Enfoque Metodol\'ogico}

Este trabajo de investigación sigue una metodología de tipo experimental y aplicada, con enfoque cuantitativo, ya que se propone diseñar, implementar y evaluar un sistema de comunicación seguro basado en el protocolo UART, incorporando funciones criptográficas mediante diseño RTL en una FPGA.




\section{Marco procedimental/ Etapas de Desarollo}
La metodología se estructura en las siguientes etapas:



\subsection{Análisis del protocolo UART }

\begin{itemize}
\item Estudio de la estructura y funcionamiento del protocolo UART.
\item Identificación de vulnerabilidades y limitaciones en cuanto a seguridad.
\end{itemize}

\subsection{Diseño de la capa criptogr\'afica }

\begin{itemize}
\item Selección de un algoritmo de cifrado simétrico adecuado (por ejemplo, AES o XOR extendido para propósitos didácticos).
\item Modelado de módulos de cifrado y descifrado a nivel RTL (Register Transfer Level).
\end{itemize}

\subsection{Integración con el protocolo UART }

\begin{itemize}
\item Modificación del módulo UART para incluir la capa criptográfica sin alterar su estructura base.
\item Definición de flujos de datos seguros: entrada cifrada, salida descifrada.
\end{itemize}

\subsection{Implementación en FPGA}

\begin{itemize}
\item Codificación en Verilog.
\item Síntesis, simulación y verificación funcional utilizando herramientas CAD (Quartus de Intel).
\item Implementación física en una FPGA (por ejemplo, una placa DE10-Lite, Nexys A7 o similar).
\end{itemize}

\subsection{ Validación y pruebas}

\begin{itemize}
\item Pruebas funcionales de transmisión segura UART en hardware.
\item Medición de rendimiento: latencia, velocidad de transmisión, utilización de recursos.
\item Comparación con versiones sin seguridad o implementaciones en software.
\end{itemize}

\subsection{ }


%\newgeometry{margin=1cm,includefoot}
%\layout
%\printinunitsof{cm}
%\pagevalues



%%%%%%fin del archivo
\endinput 