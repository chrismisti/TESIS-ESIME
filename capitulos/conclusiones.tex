
\chapter{Conclusiones}

una conclusi\'on\index{conclusión}

\section{Recomendaciones y trabajo futuro}

%
\chapter{Referencias Bibliográficas}
\begin{itemize}
\item IEEE Standards Association. (1995). IEEE Standard Verilog Hardware Description Language (IEEE Std 1364-1995). Institute of Electrical and Electronics Engineers.
\item IEEE Standards Association. (2005). IEEE Standard for SystemVerilog—Unified Hardware Design, Specification, and Verification Language (IEEE Std 1800-2005). Institute of Electrical and Electronics Engineers.
\item Palnitkar, S. (2003). Verilog HDL: A Guide to Digital Design and Synthesis (2nd ed.). Prentice Hall.
\item Edad del Hierro
\item Kahn, D. (1996). The codebreakers: The comprehensive history of secret communication from ancient times to the Internet (2nd ed.). Scribner.
\item Schneier, B. (2015). Applied cryptography: Protocols, algorithms, and source code in C (20th anniversary ed.). Wiley.
\item Singh, S. (2000). The code book: The science of secrecy from ancient Egypt to quantum cryptography. Anchor Books.
\item Stallings, W. (2017). Cryptography and network security: Principles and practice (7th ed.). Pearson.
\end{itemize}












Kahn, D. (1996). The Codebreakers: The Comprehensive History of Secret Communication from Ancient Times to the Internet. Scribner.

Stallings, W. (2017). Cryptography and Network Security: Principles and Practice (7th ed.). Pearson.

Schneier, B. (2015). Applied Cryptography: Protocols, Algorithms, and Source Code in C (20th Anniversary ed.). Wiley.

Singh, S. (2000). The Code Book: The Science of Secrecy from Ancient Egypt to Quantum Cryptography. Anchor Books.





%%%%%%fin del archivo
\endinput 

